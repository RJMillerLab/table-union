\documentclass[12pt]{article}
\begin{document}

{\bf Unionability}


Das Sarma et al. consider two factors in measuring the unionability of domains: 
entity consistency and entity expansion~\cite{DasSarma:2012}. 
In their work, entity consistency is quantified by comparing domains in ontology space. 
Domains are mapped to classes in an ontology using the semantic relatedness of their 
entity values to Classes.  

In ontology $O$, a domain $D_i =\{v_i^1, v_i^2, \ldots\}$ is represented as 
$O(D_i) =\{c_1:w_i^1, c_2:w_i^2, \ldots, c_k:w_i^k\}$, where $c_j\in Classes(O)$ and 
$w_j=f(D_i, c_j)$ is a calculated score, using some function $f$, for the relatedness of domain $D_i$ to class $c_j$. 
The ontology similarity ($OS$) of a query domain $D_q$ and a candidate domain $D_c$, 
is defined as follows: 
\begin{equation}
	SO(D_q, D_c) = Sim(O(D_q), O(D_c))
\end{equation}
Dot-product is the similarity function used by Das Sarma et al. 

$O(D_i)$ represents domain $D_i$ in the space of ontology classes. 
However, $c_i$'s are not orthogonal dimensions of the ontology. 

In embedding space, we represent domain $D_i =\{v_i^1, v_i^2, \ldots\}$ as 
$E(D_i) = \{emb(v_i^1), emb(v_i^2), \ldots\}$, where $emb(v_i)$ is the embedding of the domain value $v_i$. 
We define the topic representation of domain $D_i$ as 
$P(E(D_i)) = \{P_i^1:v_i^1, \ldots, P_i^k:v_i^k\}$, where 
$P_j$ is the $j$-th principal component of $E(D_i)$ and $v_j$ is the variance of $P_j$. 
In the embedding topic space, $P_j$'s and $v_j$'s are analogous to $c_k$'s and $w_k$'s in ontology space. 
However, $P_j$'s are data-dependent and variable for different domains, 
as opposed to $c_j$'s that are fixed for all domains. 

Assume $SE(E(D_q), E(D_c))$ is the semantic similarity of domains $D_c$ and $D_q$, in topic embedding space. 
Various similarity measures can be considered for comparing domains in topic embedding space. 
If Probabilistic PCA is used for generating principal components, 
log-likelihood of $D_c$ in the topic embedding space of $D_q$ can be used as a semantic similarity measure. 

Another way is to find the alignment $a$ between $P_i$'s in $P(E(D_c))$  
and $P_j$'s in $P(E(D_q))$ such that $\sum Cosine(a)$ is maximized. 

We show that $SE(D_q, D_c)$ is monotonically increasing with respect to 
$Cosine(nearest(P(E(D_c)), P(E(D_q))))$. 
 
The notion of entity expansion can be quantified by the containment score of 
query domain and candidate domains. 

\bibliographystyle{plain}
\bibliography{unionability}

\end{document}\documentclass[12pt]{article}
